\chapter{Аналитический раздел}
\label{cha:analysis}
%
% % В начале раздела  можно напомнить его цель
%
В данном разделе производится анализ процессов сбора информации, сохранения её на стороне сервера и распространение её средствами API.
Производится анализ подсистем, входящих в реализуемы программный комплекс, формируются требования к создаваемой системе, выделяются функции её подсистем и описывается взаимодействие между ними.

\section{Общее описание системы}
В данной работе разрабатывается система полного цикла сбора, хранения и распространения информации, для функционирования которой необходимо спроектировать все подсистемы программного комплекса и способы их взаимодействия.

Для достижения поставленной цели необходимо спроектировать и разработать следующие компоненты системы:

\begin{enumerate}
\item подсистему сбора информации, находящихся на просторах интернета в виде страниц HTML, работающую в фоновом режиме;
\item базу данных для хранения собранной информации и быстрого доступа к ней;
\item простой и удобный в использовании REST API для предоставления доступа к собранной информации сторонним приложениям;
\item клиентское приложение, используемого в качестве клиента API для демонстрации возможностей разработанной.
\end{enumerate}

Так же для удобства и простоты использования необходимо разработать доступную документацию разработанного API, для того, чтобы любой желающий разработчик имел возможность легко и быстро воспользоваться услугами сервиса. 

Поскольку конечным пользователем разрабатываемой системы будет некое стороннее приложение, рассмотрим работу клиента системы с точки зрения разработчика-пользователя API. Работа с системой выглядит следующим образом:

\begin{enumerate}
\item разработчик создает приложение, имеющее возможность доступа в интернет;
\item он регистрирует свое приложение в системе для получения доступа к API;
\item используя известный протокол прикладного уровня передачи данных в среде web, приложение обращается по определенному адресу к API;
\item при обмене сообщениями протокола запрашиваемая система(сервер) отсылает ответ клиенту;
\item клиентское приложение использует полученную информацию по своему усмотрению.
\end{enumerate}

\section{Обзор существующих решений}
В настоящее время существует некоторое количество подобных решений, предоставляющих API(REST, SOAP, XML-RPC, и т.д.) для доступа к спортивной информации сторонним приложениям.
Но они ограничены в том или ином виде. Наиболее полные и интересные решения являются платными или условно бесплатными, предоставляющими пробный, ограниченный функционал бесплатно. Бесплатные API решения предоставляют неполную информацию или плохо документированный сервис. В большинстве API отсутствует информации о российских турнирах, что является одним из решающих факторов в пользу создания собственного API.

Наиболее интересными решениями в сфере предоставления спортивной информации являются следующие сервисы:

\begin{itemize}
\item \url{http://developer.espn.com/} - API был разработан для того, чтобы быть простым в использовании. Поддерживает XML, JSON/JSONP форматы ответа сервера. Имеет хорошую документацию. К сожалению нацелен на американского разработчика, так как в основном предоставляет доступ к информации, связанной с наиболее распространенными видам спорта в америке. Имеет условно бесплатный доступ;
\item \url{http://www.footytube.com/openfooty/} - Интересный сервис предоставляющие REST API возвращающий ответ в формате XML. Предоставляет информацию только о футболе. Хорошо документирован. Имеет ограничение в 5000 запросов в день;
\item \url{http://www.championat.com/} - Популярный российский спортивный портал, содержащий большое количество статистической и исторической информации, новостей из мира спорта, информации о текущих и будущих событиях. Имеет широкий круг рассматриваемых видов спорта, охватывающий широкий круг пользователей. Имеет элементы социальной сети. К сожалению не имеет собственного API;
\item ...
\end{itemize}

Ни один из этих сервисов не предоставляет информации о спортивных событиях в режиме "live".

Таким образом, обзор существующих решений показал, что все сервисы, предоставляющие подобный функционал являются довольно узкоспециализированными, ограниченными и не универсальными, а сервисы, предоставляющие необходимые данные в полном или практически полном объеме не предоставляют прямого доступа к ним посредствам API. 

\section{Подсистема сбора данных}
\subsection{Общие представления о системе}
 % Обратите внимание, что включается не ../dia/..., а inc/dia/...
% В Makefile есть соответствующее правило для inc/dia/*.pdf, которое
% берет исходные файлы из ../dia в этом случае.

Подсистема сбора данных должна решать проблемы "переваривания" "сырых" данных из сети и формирования из них базы данных. Таким образом, собранная на просторах интернета информация должна принять удобный для хранения, использования и распространения, структурированный вид. Данных процесс в общем случае представлен на рисунке 1.1.
\begin{figure}
  \centering
  \includegraphics[width=\textwidth]{inc/dia/analysis1-1}
  \caption{Рисунок}
  \label{fig:fig01}
\end{figure}


\subsection{Существующие сервисы сбора информации}

\begin{itemize}
\item \url{http://www.mozenda.com/} - сервис сбора информации в интернете. Является SaaS-приложением для решения схожих проблем поиска информации в сети. Является довольно дорогим сервисом. Предоставляет on-line конструктор web-пауков для сбора информации, что является несомненным плюсом сервиса
\item \url{http://www.fetch.com/} - название сайта говорит само за себя. Является похожим сервисом сбора информации с просторов паутины, но не предоставляет полной информации о своем сервисе и условиях использования на сайте. Для полного ознакомления предлагает связаться со службой поддержки.
\end{itemize}

Приведенный список показывает, что сбор информации в интернете является популярной и сложной задачей. Для универсального решения данной задачи требуется спроектировать сложный программно-аппаратный комплекс, требующий огромных вычислительных мощностей, что выходит за рамки данной работы и может воплотиться в виде развития рассматриваемой темы. 

\subsection{Процесс сбора информации}

Подсистема сбора информации должна работать в режиме 24/7 на некотором сервере и должна предоставлять возможность сбора информации сразу с нескольких предполагаемых источников.
Система сбора должна быть полностью автоматизирована и получать от администратора только правила фильтрации данных. Так как данная система работает в сети интернет, она должна быть готова к "отказам" серверных участников обмена информацией(источников данных), то есть при невозможности выполнения задачи получения страницы, система должна отложить данную задачу либо немедленно повторить процесс. Поскольку системе задаются некоторые правила сбора, она должна фиксировать промежуточные результаты работы в некотором хранилище. Такой подход позволит провести декомпозицию процесса сбора на элементарные единицы работы, что позволит более гибко реагировать на отказы системы, нештатные ситуации, связанные с работой сети и другие возможные сложности при выполнения сбора единицы полезной информации.

Данные требования хорошо решаются путем создания очереди заданий, представляющих собой систему массового обслуживания. Данная система состоит из очереди задач и обслуживающего аппарата. Поскольку процесс сбора единицы информации проходит декомпозицию на подзадачи, данные подзадачи должны ставиться на обслуживание в очередь отдельно. Таким образом мы получаем очередь с разными типами заявок. В случае простого накопления данных очередность выполнения заданий не имеет значения, так как итоговой целью работы очереди является выполнение всех задач. В нашем случае некоторые задачи могут иметь более высокий приоритет, так как должны быть доступны сразу после момента поступления(с некоторым допущением).
Данное требование естественным образом вводит свойство приоритета задачи. Таки образом, мы получаем очередь с приоритетами и, поскольку, данная работа не требует немедленного выполнения(из-за того же допущения, связанного со спецификой распространения информации в сети), то допускается ожидание завершения выполнения заявки, находящейся на обслуживании.
Данные требования приводят нас к решению использования очереди заданий с относительными приоритетами. 

На более низком уровне процесс сбора единицы информации выглядит следующим образом:
\begin{enumerate}
\item Система переходит по некоторой базовой ссылке, ведущей на определенный ресурс в интернете;
\item Полученный ответ система пытается разобрать по некоторым, заранее заданным правилам;
\item В случае нахождения очередной ссылки, в очередь ставится новая задача;
\item Процесс продолжается до тех пор, пока выполнение разбора не приведет к итоговой цели единицы информации;
\item В случае сбоя в процессе выполнения задачи, она ставится в очередь для повторного выполнения;
\end{enumerate}

Данный процесс представлен на рисунке 1.2.
Подсистема сбора данных должна решать проблемы "переваривания" "сырых" данных из сети и формирования из них базы данных. Таким образом, собранная на просторах интернета информация должна принять удобный для хранения, использования и распространения, структурированный вид. Данных процесс в общем случае представлен на рисунке 1.1.
\begin{figure}
  \centering
  \includegraphics[width=\textwidth]{inc/dia/analysis1-2}
  \caption{Рисунок}
  \label{fig:fig02}
\end{figure}
