\chapter{Охрана труда и экология}

\section{Оценка условий труда на рабочем месте пользователя ПЭВМ}

Разработка ПО требует длительного взаимодействия с вычислительными системами. Работа с ПЭВМ связана с рядом вредных и опасных факторов, таких как статическое электричество, рентгеновское излучение, электромагнитные поля, блики отраженный свет, ультрафиолетовое излучение. При длительном воздействии на организм эти факторы негативно влияют на здоровье человека.

\subsection{Параметры микроклимата}

Параметры микроклимата могут меняться в широких пределах, в то время как необходимым условием жизнедеятельности человека является поддержание постоянства температуры тела благодаря терморегуляции, т.е. способности организма регулировать отдачу тепла в окружающую среду. Принцип нормирования микроклимата - создание оптимальных условий для теплообмена
тела человека с окружающей  средой. Вычислительная техника является источником существенных тепловыделений, что может привести к повышению температуры и снижению относительной влажности в помещении. В помещениях, где установлены компьютеры, должны соблюдаться определенные параметры микроклимата. Нормы, установленные СанПиН 2.2.2/2.4.1340-03 для категории работ 1а приведены в таблице~\ref{tab:microclimate}. Эти нормы устанавливаются в зависимости от времени года, характера трудового процесса и характера производственного помещения.

\begin{table}[ht]
\caption{Параметры микроклимата}
\begin{tabular}{|l|c|c|c|c|c|c|}
\hline
\multirow{2}{*}{Период год} & \multicolumn{2}{l|}{Температура, $^\circ \mbox{C}$} & \multicolumn{2}{l|}{Влажность, \%} & \multicolumn{2}{l|}{Скорость воздуха, м/с} \\
\cline{2-7}
&Оптим.&Допуст.&Оптим.&Допуст.&Оптим.&Допуст.\\
\hline
Холодный &22--24&21--25&40--60&75&0.1&0.1\\
\hline
Теплый &23--25&22--28&40--60&55 при 28$^\circ \mbox{C}$&0.1&0.1\\
\hline 
\end{tabular}
\label{tab:microclimate}
\end{table}

Вредным фактором при работе с ЭВМ является также запыленность помещения. Этот фактор усугубляется влиянием на частицы пыли электростатических полей персональных компьютеров.

Для устранения несоответствия параметров указанным нормам проектом предусмотренно использование системы кондиционирования как наиболее эффективного и автоматически функционирующего средства.

Нормы установленные содержания в воздухе положительных и отрицательной ионов, установленные СанПиН 2.2.4.1294--03, приведены в таблице~\ref{tab:ions}.

\begin{table}[ht]
\caption{Уровни ионизации воздуха при работе на ПЭВМ}
\begin{tabular}{|l|c|c|}
\hline
\multirow{2}{*}{Уровни} & \multicolumn{2}{l|}{Число ионов в кубометре воздуха}\\
\cline{2-3}
&$n^+$&$n^-$\\
\hline
Минимально необходимое & 400 & 600 \\
\hline
Оптимальное & 1500--3000 & 3000--5000 \\
\hline
Максимально допустимое & 50000 & 50000 \\
\hline
\end{tabular}
\label{tab:ions}
\end{table}

Для обеспечения требуемых уровней предусмотренно использование системы ионизации Сапфир-4А.

Объем помещений, в которых размещены работники вычислительных центров, не должен быть меньше $19.5 \frac{\text{м}^3}{\text{человека}}$ с учетом максимального числа одновременно работающих в смену. Нормы подачи свежего воздуха в помещения, где расположены компьютеры, приведены в таблице~\ref{tab:air}.

\begin{table}[ht]
\caption{Уровни ионизации воздуха при работе на ПЭВМ}
	\begin{tabular}{|l|l|}
	\hline
	Уровни & Число ионов в кубометре воздуха \\ 
	\hline
	Объем до $20 {\text{м}^3}$ на человека & Не менее 30 \\
	\hline
	$20.40 {\text{м}^3}$ на человека & Не менее 20 \\
	\hline
	Более $40 {\text{м}^3}$ на человека & Естественная вентиляция \\
	\hline
	\end{tabular}
\label{tab:air}
\end{table}

Для обеспечения комфортных условий используются как организационные методы (рациональная организация проведения работ в зависимости от времени года и суток, чередование труда и отдыха), так и технические средства (вентиляция, кондиционирование воздуха, отопительная система).

\subsection{Шум и вибрации}

Уровень шума на рабочем месте программиста не должен превышать 50 дБА, а уровень вибрации не должен превышать норм установленных СанПиН 2.2.2.542--96 (см. таблицу~\ref{tab:vibro}).

\begin{table}[ht]
\caption{Допустимые нормы вибрации на раочих местах с ПЭВМ}
\begin{tabular}{|c|c|c|}
\hline
\parbox{0.4\textwidth}{ Среднегеометрические частоты\\октавных полос, Гц}& \multicolumn{2}{l|}{Допустимые значения по виброскорости}\\
\cline{2-3}
&м/c &дБ\\
\hline
2  & $4.5\times10$ & 79 \\
\hline
4  & $2.2\times10$ & 73 \\
\hline
8  & $1.1\times10$ & 67 \\
\hline
16  & $1.1\times10$ & 67 \\
\hline
31.5 & $1.1\times10$ & 67 \\
\hline
63  & $1.1\times10$ & 67 \\
\hline
\parbox{0.4\textwidth}{ Корректированные значения\\и их уровни в дБ}& $2.0\times10$ & 72\\
\hline
\end{tabular}
\label{tab:vibro}
\end{table}

При разработке ПО внутренними источниками шума являются вентиляторы, а также принтеры и другие перефферийные устройства ЭВМ. Внешние источники шума~--- прежде всего, шум с улицы и из соседних помещений. Постоянные внешние источники шума, превышающего нормы, отсутствуют.

Для устранения превышения нормы проектом предусмотрено применение звукопоглощающих материалов для облицовки стен и потолка помещения, в котором осуществляется работа с вычислительной техникой.

\subsection{Освещение}

Наиболее важным условием эффективной работы программистов и пользователей является соблюдение оптимальных параметров системы освещения в рабочих помещениях.

Естественное освещение осуществляется через светопроемы, ориентированные в основном на север и северо-восток (для исключения попадания прямых солнечных лучей на экраны компьютеров) и обеспечивает коэффициент естественной освещенности (КЕО) не ниже 1.5\%.

В качестве искусственного освещения проектом предусмотрено использование системы общего освещения. в соответствии с СанПин 2.2.2/2.4.1340--03 освещенность на поверхности рабочего стола должна находиться в пределах 300--500 лк. Разрешается использование светильников местного освещения для работы в документами (при этом светильники не должны создавать блики на поверхности экрана).

Правильное расположение рабочих мест относительно источников освещения, отсутствие зеркальных поверхностей и использование матовых материалов ограничивает прямую (от источников освещения) и отраженную (от рабочих поверхностей) блескость. При  этом яркость светящихся поверхностей не превышает $200 \frac{\text{кд}}{\text{м}^2}$, яркость бликов на экране ПЭВМ не превышает $40 \frac{\text{кд}}{\text{м}^2}$, и яркость потолка не превышает $200 \frac{\text{кд}}{\text{м}^2}$.

В соответствии с СанПинН 2.2.2/2.4.1340--03 проектом предусмотрено использование люминесцентных ламп типа ЛБ в качестве источников света при искусственном освещении. В светильниках допускается применение ламп накаливания. Применение газоразрядных ламп в светильниках общего и местного освещения обеспечивает коэффициент пульсации не более 5\%.

Таким образом, проектом обеспечиваются оптимальные условия освещения рабочего помещения.

\subsection{Рентгеновское излучение}

В соответствии с СанПиН 2.2.2/2.4.1340-03 проектом предусмотрено использование ПЭВМ, конструкция которого обеспечивает мощность экспозиционной дозы рентгеновского излучения в любой точке на расстоянии 0.5 м. от экрана и корпуса не более 0.1 мбэр/час (100 мкР/час). Результаты сравнения норм излучения приведены в таблице~\ref{tab:rentgen}.

\begin{table}[ht]
\caption{Сравнение норм рентгеновского излучения в различных стандартах}
\begin{tabular}{|l|c|}
\hline
& Допустимое значение мкР/час, не более \\
\hline
СанПиН 2.2.2/2.4.1340-03 & 100 \\
\hline
ТСО-99 & 500 \\
\hline
MPR II & 500\\
\hline
\end{tabular}
\label{tab:rentgen}
\end{table}

Как видно из таблицы, стандарты MPR II и ТСО--99 предъявляют менее жесткие требования к рентгеновскому излучению, чем СанПиН. Но при соблюдении оптимального расстояния между пользователем и монитором дозы рентгеновского излучения не опасны для большинства людей.

\subsection{Неионизирующие электромагнитные излучения}

Допустимые значения параметров неионизирующих излучений в соответствии с СанПин 2.2.2/2.4.1340-03 приведены в таблицах~\ref{tab:U} и~\ref{tab:ro}.

\begin{table}[ht]
\caption{Предельно допустимые значения напряженности электрического поля}
\begin{tabular}{|c|c|}
\hline
Диапазон частот& Допустимые значения \\
\hline
5 Гц -- 2 кГц & 25 В/м \\
\hline
2 -- 400 кГц& 2.5 В/м \\
\hline
\end{tabular}
\label{tab:U}
\end{table}

\begin{table}[ht]
\caption{Предельно допустимые значения плотности магнитного потока}
\begin{tabular}{|c|c|}
\hline
Диапазон частот& Допустимые значения \\
\hline
5 Гц -- 2 кГц & 250 нТл \\
\hline
2 -- 400 кГц& 5 нТл \\
\hline
\end{tabular}
\label{tab:ro}
\end{table}

Величина поверхностного электрического потенциала не должна превышать 500 В.

Мониторы, используемые в настоящее время, удовлетворяют более жестким нормам MPR II, а значит и СанПиН.

\subsection{Визуальные параметры}

Неправильный выбор визуальных эргономических параметров приводит к ухудшению здоровья пользователей, быстрой утомляемости, раздражительности. В связи с этим, проектом предусмотрено, что конструкция вычислительной системы и ее эргономические параметры обеспечивают комфортное и надежное считывание информации. Требования к визуальным параметрам, их внешнему виду, дизайну, возможности настройки представлены в СанПиН 2.2.2/2.4.1340--03. Визуальные эргономические параметры монитора и пределы из изменений приведены в таблице~\ref{ergonom}.

\begin{table}[ht]
\caption{Визуальные эргономические параметры ВДТ и пределы из изменений}
\begin{tabular}{|l|c|c|}
\hline
\multirow{2}{*}{Наименование параметров} & \multicolumn{2}{c|}{Пределы значений параметров}\\
\cline{2-3}
&не менее&не более\\
\hline
Яркость экрана (фона), $\frac{\text{кд}}{\text{м}^2}$ (измеренная в темноте) &35&120\\
\hline
Внешняя освещенность экрана, лк &100&250\\
\hline
Угловой размер экрана, угл.мин. &16&60\\
\hline
\end{tabular}
\label{tab:ergonom}
\end{table}

Для выполнения этих требований проектом предусмотренно использование современных мониторов, имеющих достаточно широкий набор регулируемых параметров.  В частности, для удобного считывания информации реализована возможность настройки положения монитора по горизонтали и вертикали. Мониторы оснащены специальными устройствами и средствами настройки ширины, высоты, яркости, контраста и разрешения изображения. кроме того, в современных мониторах зерно изображения имеет размер в пределах 0.27 мм, что обеспечивает высокую четкость и непрерывность изображения. Наконец, на поверхность дисплея нанесено матовое покрытие, чтобы избавиться от солнечных бликов.

\section{Расчет искусственного освещения}

При расчете освещенности от светильников общего равномерного освещения наиболее часто применяют метод расчета по световому потоку. При расчете освещения по этому методу необходимое количество светильников для освещения рабочего места рассчитывается по формуле:

\begin{equation}
\label{f:lightsCount}
N = \frac{E_{min}\cdot S\cdot K}{F_\text{Л} \cdot \text{З} \cdot z \cdot h}
\end{equation}

где $E_{min}$~--- нормируемая минимальная освещенность; $S$~--- площадь помещения, $\text{м}^2$; $F_\text{Л}$~--- световой поток лампы, лк; $K$~--- коэффициент запаса; $z$~--- коэффициент неравномерности освещения (для люминесцентных ламп~---1.1); $h$~--- коэффициент использования светового потока в долях единицы.

$E_{min}$ определяется на основании нормативного документа СНиП23--05--95. В соответствии с произведенным выбором в предыдущем разделе, для работы программиста $E_{min}=300$ лк (общее освещение).

Работы, производятся в помещении, требуют различения цветных объектов при невысоких требованиях к цветоразличению, поэтому в качестве источника освещения была выбрана лампа люминесцентная холодно-белая (ЛХБ), 1940 лк, 30 Вт. В помещениях общественных и жилых зданий с нормальными условиями среды: К=1.4.

Для люминесцентных ламп коэффициент неравномерности освещения Z=1.1.

Коэффициент использования h зависит от типа светильника, от коэффициентов отражения потолка $\rho_\text{П}$, стен $\rho_\text{С}$, расчетной поверхности $\rho_\text{Р}$ и индекса помещения.

Высота подвеса над рабочей поверхностью Нр=3 м. Размеры помещения А=3.5 м, В=3 м. Определим индекс помещения по формуле:

\begin{equation}
\phi = \frac{A \cdot B}{H_P \cdot (A + B)} = \frac{3.5 \cdot 3}{s \cdot (3.5 + 3)} = 0.54
\end{equation}

Для светлого фона примем:$\rho_\text{П} = 70$ $\rho_\text{С} = 50$ $\rho_\text{Р} = 10$. h = 59\%.

Освещение проектируется при помощи светильников ОДОР с минимальной освещенностью $E_{min}=300$ лк, P=40 Вт. Число ламп в ОДОР равно 2. Необходимое число светильников для данной комнаты:

\begin{equation}
N = \frac{300 \cdot 9 \cdot 1.4}{1940 \cdot 0.59 \cdot 1.1 \cdot 2} = 2 \text{шт}
\end{equation}

Общее количество ламп $n = 2\times2=4$ шт. Длина светильника ОДОР=1.26 м. Поскольку длина помещения 3 м, то светильники помещаются в два ряда. 
Суммарная мощность светильников: $30\cdot4=160$ Вт. Сумарный световой поток: $1940\cdot4=7760$ лм.

\section{Режим труда}
При работе с персональным компьютером очень важную роль играет соблюдение правильного режима труда и отдыха. В противном случае у программиста отмечаются значительное напряжение зрительного аппарата с появлением жалоб на неудовлетворенность работой головные боли, раздражительность, нарушение сна, усталость и болезненные ощущения в глазах, в пояснице, в области шеи и руках.

При несоответствии фактических условий труда требованиям санитарных правил и норм время регламентированных перерывов следует увеличить на 30\%. В соответствии со СанПиН 2.2.2 546-96 все виды трудовой деятельности, связанные с использованием компьютера, разделяются на три группы: 
\begin{enumerate}
	\item группа А: работа по считыванию информации с экрана ВДТ или ПЭВМ с предварительным запросом; 
	\item группа Б: работа по вводу информации; 
	\item группа В: творческая работа в режиме диалога с ЭВМ. 
\end{enumerate}

Режим труда и отдыха должен зависеть от характера работы: при вводе данных, редактировании программ, чтении информации с экрана непрерывная продолжительность работы с монитором не должна превышать 4 часов. При 8 часовом рабочем дне, через каждый час работы необходимо проводить перерыв 5--10 минут, а каждые два часа перерыв в 15 мин.

Эффективность перерывов повышается при сочетании с производственной гимнастикой или организации специального помещения для отдыха персонала с удобной мягкой мебелью, аквариумом, зеленой зоной и т.п.
