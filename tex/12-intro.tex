\Introduction

В настоящее время всемирная паутина предоставляет огромное количество информации для пользователя на любой вкус. Миллиарды байт буквально лежать на кончиках пальцев миллионов пользователей. И в наше время не обязательно быть привязанным к компьютеру, чтобы получить доступ к этой информации. Достаточно иметь телефон, планшет или даже фотоаппарат с поддержкой беспроводной технологии доступа в интернет и ты получаешь возможность получить любую информацию за пару прикосновений прохладного стекла твоего девайса. Но протокол HTTP, благодаря которому мы имеем сегодня поистине безграничные возможности, разработан, как говорит название, для передачи гипертекста, то есть, данные приходят клиенту в виде размеченного документа. Это очень удобно при чтении документа с экрана компьютера или другого устройства отображения информации человеку, но совершенно не приемлемо для использования в других автоматизированных системах, когда клинтом является другое приложения. В таком случае язык HTML, который в основном и используется для распространения информации во всемирной паутине, является декоратором необходимых данных. 

При разработке приложений разработчики сталкиваются в первую очередь с проблемой получения и хранения информации. Первая проблема решается путем использования специализированных интерфейсов для получения данных и заботливые разработчики начинают все больше внимания уделять разработке API для того, чтобы их сервисы были удобнее как для простых пользователей, так и для программистов, желающих использовать накопившуюся у них ценную информацию в собственных целях.

Но так бывает далеко не во всех случаях, и в такой ситуации разработчику приходится разгребать кучи HTML кода для фильтрации необходимых данных и отсеивания лишней разметки для того, чтобы превратить тонны неструктурированной информации с просторов паутины в четко структурированные единицы данных. В таком случае очень удобна автоматизация процесса накопления, фильтрации и дальнейшего сохранения необходимой информации на стороне сервиса. 

Целью дипломной работы является создание программного комплекса для предоставления справочной информации для разработчиков о спортивных мероприятиях. Для достижения данной цели необходимо выполнить следующее: 

\begin{enumerate}
\item разработать подсистему сбора информации, распространяемой в сети Internet в неструктурированном виде, в формате HTML страниц, работающую в реальном времени;
\item спроектировать базу данных для хранения собранной информации;
\item спроектировать REST API для предоставления доступа к собранной информации в удобном для использования виде;
\item проверить работоспособность системы путем создания приложения, используемого в качестве клиента API.
\end{enumerate}

Подсистема сбора информации должна работать в качестве "демона" на сервере и представлять из себя очередь заданий для выполнения задач сбора. Собранная информация должна сохраняться в спроектированной базе данных через API для большей надежности и безопасности системы. Так же API должно предоставлять доступ на чтение для сторонних разработчиков, заинтересованных в хранимой информации. Клиентское приложение должно представлять собой web-сайт, предоставляющий пользователю доступ к статистической и исторической информации о спортивных мероприятиях. 

В данной работе упор делается на футбольную статистику, но разрабатываемая система должна быть легко расширяем для сбора и сохранения информации по другим видам спорта. Таким образом, каждая подсистема должна быть гибка и открыта для расширения.
